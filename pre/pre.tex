\documentclass[aspectratio=169]{ctexbeamer} % normal mode:& for lecturing 
% \documentclass[aspectratio=169, draft]{ctexbeamer} % draft mode:& to compile faster 
% \documentclass[aspectratio=169, handout]{ctexbeamer} % handout mode:& when releasing slides 
% \setbeamertemplate{footline}[frame number] % use this when releasing pdf
\usepackage{amsmath}
\usepackage{booktabs}
\usepackage{graphicx}
\usepackage{ulem}
\usepackage{xcolor}
\setbeamertemplate{caption}{\raggedright\insertcaption\par}
% \setbeamercovered{transparent}

\AtBeginSection[]
{
\begin{frame}
    \frametitle{}
    \tableofcontents[currentsection]
\end{frame}
}
% TODO: fill title here
\title{Final project: 用两个三角形渲染世界}
\author{张高阳}
% \date{Feb 23 2020}

\begin{document}
\begin{frame}
    \titlepage 
\end{frame}

\begin{frame}
    \frametitle{简介}
    \begin{columns}
        \begin{column}{.1\textwidth}\end{column}
        \begin{column}{.5\textwidth}
            \begin{itemize}
                \item 选题: \underline{2.4.2}: 用两个三角形渲染世界
                \item 渲染物体: 冰淇淋 \texttt{emoji}
            \end{itemize}
        \end{column}
        \begin{column}{.4\textwidth}
            \includegraphics[width=32pt]{images/pre/emoji.pdf}
        \end{column}
    \end{columns}
\end{frame}

\begin{frame}
    \frametitle{效果}
    \begin{figure}[htbp]
        \centering
        \includegraphics[height=.85\textheight]{images/pre/full.pdf}
        \caption{}
        \label{fig:full}
    \end{figure}
\end{frame}

\begin{frame}
    \frametitle{基本思想}
    \begin{itemize}
        \item 用两个直角三角形覆盖整个屏幕
        \item 在片段着色器中构建距离场 (signed distance function) % 法线方向估计, 物体内部距离为负
        \item 使用 ray march 方法逐像素渲染 % adaptive epsilon
    \end{itemize}
    \begin{figure}[htbp]
        \centering
        \includegraphics[height=.5\textheight]{images/pre/iq_2tri.pdf}
        \caption{\footnotesize 图片来自 \url{https://www.iquilezles.org/www/material/nvscene2008/rwwtt.pdf}}
        \label{fig:iq_2tri}
    \end{figure}
\end{frame}

\begin{frame}
    \frametitle{目录}
    \tableofcontents
\end{frame}

% TODO: slides 
%************************%
% TODO: fill section (subsection, subsubsection) title here
\section{实现过程}
\subsection{甜筒}
% \subsubsection*{}
\begin{frame}
    \frametitle{分解} % frame title
    \begin{columns}
    \begin{column}{.1\textwidth}\end{column}
        \begin{column}{.4\textwidth}
            \begin{itemize}
                \item 圆台 $\times 2$
                \item 碗形 $\times 1$
                \item 环 $\times 3$
                \item 竖线 $\times 7$
            \end{itemize}
        \end{column}
        \begin{column}{.5\textwidth}
            \includegraphics[width=128pt]{images/pre/emoji.pdf}
        \end{column}
    \end{columns}
\end{frame}
\begin{frame}
    \frametitle{圆台} % frame title
    \begin{columns}
    % \begin{column}{.1\textwidth}\end{column}
        \begin{column}{.6\textwidth}
            \begin{itemize}
                \item 以 Y 轴为中心对称
                \item \texttt{vec3 q = vec3(length(p.xz), p.y)}
                \item 梯形
            \end{itemize}
        \end{column}
        \begin{column}{.4\textwidth}
            \begin{figure}[htbp]
                \centering
                \includegraphics[height=.65\textheight]{images/pre/cone_derive.pdf}
                \caption{}
                \label{fig:cone_derive}
            \end{figure}
        \end{column}
    \end{columns}
\end{frame}
\begin{frame}
    \frametitle{碗形} % frame title
    \begin{columns}
    % \begin{column}{.1\textwidth}\end{column}
        \begin{column}{.6\textwidth}
            \begin{itemize}
                \item \texttt{vec3 q = vec3(length(p.xz), p.y)}
                \item 圆 (部分)
            \end{itemize}
        \end{column}
        \begin{column}{.4\textwidth}
            \begin{figure}[htbp]
                \centering
                \includegraphics[height=.65\textheight]{images/pre/bowl_derive.pdf}
                \caption{}
                \label{fig:bowl_derive}
            \end{figure}
        \end{column}
    \end{columns}
\end{frame}
\begin{frame}
    \frametitle{五个部分}
    \begin{figure}[htbp]
        \centering
        \includegraphics[height=.33\textheight]{images/pre/cone/cone1.pdf}
        \includegraphics[height=.33\textheight]{images/pre/cone/cone2.pdf}
        \includegraphics[height=.33\textheight]{images/pre/cone/bowl.pdf}
        \includegraphics[height=.33\textheight]{images/pre/cone/rings.pdf}
        \includegraphics[height=.33\textheight]{images/pre/cone/bars.pdf}
        \caption{}
        \label{fig:cone_parts}
    \end{figure}
\end{frame}
\begin{frame}
    \frametitle{组装}
    \begin{figure}[htbp]
        \centering
        \includegraphics[height=.75\textheight]{images/pre/cone/final.pdf}
        \caption{甜筒模型}
        \label{fig:cone_model}
    \end{figure}
\end{frame}
%************************%
\subsection{奶油}

\subsection{糖果}
\subsection{环境光照}
\subsection{软阴影}
\subsection{材质}

\begin{frame}
    \begin{center}
        \Huge{感谢观看!}
    \end{center}
\end{frame}

\end{document}